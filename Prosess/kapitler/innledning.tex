\documentclass[../main.tex]{subfiles}

\chapter{Innledning}

Vi er fem studenter som har jobbet sammen gjennom dette semesteret i forbindelse med EIT(TTK4850), byggelandsbyent for å etablere en studentorganisasjon samt bygge en fjernstyrt undervannsrobot.

Vi kommer alle fra fire forskjellige studieretninger, vi har ulike erfaringer med gruppearbeid og ulik personlig kompetanse.

Denne rapporten skal fortelle noe om hvilke forventinger hver enkelt hadde før vi begynte, hvordan vi som gruppe har utviklet oss og hva vi har erfart og lært både om oss selv i grupper og generelt om ulike gruppeprosesser som har oppstått underveis. Vi vil også vise hvordan vi har greid å kombinere og utnytte den ulike kompetansen, både faglig og personlig, til hver enkelt i gruppen.

Dette gjør vi ved først å begynne med individuelle presentasjoner av hver enkelt i gruppen. Deretter vil vi beskrive tre ulike situasjoner og diskutere rundt de prosessene som oppstod da. Her kobler vi det sammen med teori om gruppedynamikk og refleksjoner om hvorfor situasjonene oppstod og endte slik de gjorde.

Som en avslutning på rapporten vil vi inkludere en felles grupperefleksjon som ser på hvordan gruppen utviklet seg underveis i arbeidet.