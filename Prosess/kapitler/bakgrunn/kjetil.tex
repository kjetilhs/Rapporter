\documentclass[../main.tex]{subfiles}

\section{Kjetil}
Kjetil er 23 år, og studere sitt fjerde år ved kybernetikk og robotikk med spesialisering innen navigasjon og fartøystyring.
\\ \\
Forventinger han hadde til faget var å se hvordan han selv fungerer i et prosjekt, og hvordan han bli bedre til å kommunisere med andre som ikke har samme fag bakgrunn som han selv. Han hadde Byggelandsbyen som førstevalg for EIT, siden han ville være i en landsby med høyt motiverte personer. Siden man måtte søke seg inn til byggelandsbyen, samt at man fikk muligheten til å bygge noe var det landsbyen som fristet mest. 
\\ \\
I løpet av EIT har han lært mer om tverfaglig kommenikasjon, og utfordringer knyttet til å finne praktiske løsninger på et teoretisk problem. Han har fått større innsikt i hvordan han arbeider i en gruppe. Han har en tendens til å bli veldig opptatt med det han holder på med. Har også funnet ut at det å stille og bli stilt kritiske spørsmål om det som en holder på med kan bidra til finne problemer som gruppen ikke er klar over.
\\ \\
Utfordringer han har møtt har vært knyttet til finne løsninger på faglige problemer i samarbeid med gruppemedlemmer med annen studiebakgrunn, så vel som takle stresse som kommer med et prosjekt av denne størresleordenen. En annen utfordring han har hatt er å kommunisere hvilke problemer han har støt på, og hvordan han har gått frem for å løse de. 
\\ \\
Han har blitt bedre til å ikke spore av i samtaller. Han har lært mer om programering av en mikrokontroller, samt hvordan løse problemer med begrenset kunnskap om problemet. Han har lært mer om å reflektere over situasjoner som oppstår og diskuterer det med de andre i gruppen.

