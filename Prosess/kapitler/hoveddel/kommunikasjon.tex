\documentclass[../main.tex]{subfiles}

\chapter{Kommunikasjon}
	\label{c:kom}

Det ble gjort mange vedtak i løpet av prosjektet, men ofte var det manglende vilje til å følge opp tiltakene. Et eksempel på dette var hvordan intern kommunikasjon skulle gjennomføres. Det ble opprettet en Facebook-gruppe første dag av prosjektet for uformell kommunikasjon. Morten og Geir hadde tidligere god erfaring med Slack og Trello i større prosjekter, og anbefalte gruppe å ta dette i bruk. Det ble gjort et vedtak på å ta i bruk Slack for offisiell kommunikasjon og diskusjoner.
\\ \\
Slack fungerer slik at man har mange tråder der hver tråd har sitt tema. Dette gir fordelen at man kan snakke åpent med mange tråder samtidig. Den har også god integrering mot Google Drive og andre plattformer som er hensiktsmessige å bruke i et stort prosjekt. Dette er funksjonalitet som Facebook i stor grad mangler. Slack er også gratis og har også god integrering imot alle plattformer. Trenger vel ikke å si hvordan slack fungerer?
\\ \\
Selv om gruppen gjorde et vedtak om å bruke Slack fortsatte man å spre informasjon via Facebook. Dette var ikke bare et problem fordi Facebook ikke har i nærheten av funksjonaliteten eller integreringen til Slack, men hovedsakelig fordi Morten ikke sjekket Facebooken sin ofte, og mistet derfor viktig informasjon. Dette var ikke hans feil, men heller resten av gruppa som brukte en kanal som ikke var vedtatt( kan kanskje heller referere til beslutningsproblemene som vi hadde. Det er nevnt i vudering av lederskap). Facebook er kun egnet til å gi ut informasjon på hendelser som skjer snart, og er ikke strukturert rundt det å finne gammel informasjon eller egnet til å gi ut detaljert informasjon. Dette er en av grunnene til at det ble brukt mye tid på møtene til å orientere, og man hadde i liten grad oversikt over hvem som gjorde hva eller den overordnede planen for prosjektet. Dette var også en av flere grunner til at møtene ble ustrukturerte og inneffektive, som er beskrevet i kapittel (HELGE). Det å få alle over på samme kommunikasjonskanal, samt sørge for at alle har relevant informasjon er noe som er essensielt for å oppnå gode resultater for et gruppearbeid \cite{Kommunikasjon}.
\\ \\
Evaluering:
Det ble tidlig klart at dette var et problem, men ingen tok på seg ansvaret for å løse problemet eller tok eierskap til løsningen (teori?). Dette er antagelig fordi vi hadde rullerende leder, med uklare ansvarsområder, som beskrevet i kapittel \ref{beslutning}.
\\ \\
En annen faktor som var med å forsterke problemet var at vi ikke kjente hverandre for godt i begynnelsen. Dette gjorde at tilbakemeldingene ofte var vage og det faktiske budskapet var tidvis godt innpakket. Etter hvert som vi ble bedre kjent, og så hvordan hvert gruppemedlem håndterte tilbakemeldinger og kritikk ble det også lettere for gruppemedlemmer å påpeke og korrigere ovenfor dem som ikke fulgte opp vedtak.
\\ \\ 
Det ble bestemt at gruppen skulle gå over til en mail løsning. Geir fikk ansvaret for å sette opp tjenesten, noe som sørget for at en fikk eierskap til problemet og ansvaret for gjennomføringen.
Problemet vedvarte, og ble først løst etter Daniel ble valgt til fast leder. Han gav Geir ansvar om å tvinge alle over på epost. Denne løsningen ble valgt fordi gruppa var vant til å bruke mediet, og alle hadde gitt opp ideen om å få alle på Slack. Geir fikk nå eierskap til beslutningen og ansvar for at den ble gjennomført. Denne løsningen var noe vi ble enige om etter å ha lest om løsningen i \cite{losning}. Epostlister ble opprettet og alle gruppemedlemmer som ikke svarte på mail, eller distribuerte viktig informasjon via Facebook eller andre uformelle kanaler. Epost er også veldig effektivt til gi riktig informasjon til kun dem det gjelder.
\\ \\
Denne løsningen var ikke perfekt og problemet med at det var vanskelig å holde oversikt over prosjektet vedvarte. Daniel tok derfor initiativ til at alle måtte dokumentere alt de gjorde, og hadde planer om å gjøre, på Google Drive.
\\ \\ 
Alt ble laget i et Google Sheet der alle planene ble fylt inn, og Daniel fulgte opp og sørget for at dokumentet hele tiden var oppdatert. Selv om det tok tid å få en endelig løsning har den fungert relativt godt. Etter flere samtaler med Spark NTNU har vi blitt guidet til å gå over på en plattform som har støtte for en større organisasjon. Vi ser også at dette er nødvendig, og blitt anbefalt å bruke Trello for å fordele oppgaver, og se hva som er gjort ferdig. Det blir da viktig å ta med seg erfaringene som har blitt gjort når en slik overgang skal gjøres til neste år.

