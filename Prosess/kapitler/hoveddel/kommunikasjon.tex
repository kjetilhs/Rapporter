\documentclass[../main.tex]{subfiles}

\chapter{Kommunikasjon og handlingssvakhet}
	\label{c:kom}

Det ble gjort mange vedtak i løpet av prosjektet, men ofte var det manglende vilje til å følge opp tiltakene. Et eksempel på dette var hvordan intern kommunikasjon skulle gjennomføres. Det ble opprettet en Facebook-gruppe første dag av prosjektet for uformell kommunikasjon. Morten og Geir hadde tidligere god erfaring med Slack og Trello i større prosjekter, og anbefalte gruppe å ta dette i bruk. Det ble gjort et vedtak på å ta i bruk Slack for offisiell kommunikasjon og diskusjoner.
  \\ \\
Selv om gruppen gjorde et vedtak om å bruke Slack fortsatte man å spre informasjon via Facebook. Som nevnt i kapittel \ref{beslutning} hadde gruppen beslutningsproblemer, samt viljen til å følge opp beslutninger. Facebook ble ikke brukt av alle, noe som medførte at informasjon ikke nådde alle i tide. Dette er en av grunnene til at det ble brukt mye tid på møtene til å orientere, og man hadde i liten grad oversikt over hvem som gjorde hva eller den overordnede planen for prosjektet. Dette var også en av flere grunner til at møtene ble ustrukturerte og inneffektive, som er beskrevet i kapittel \ref{c:moter}. Det å få alle over på samme kommunikasjonskanal, samt sørge for at alle har relevant informasjon er noe som er essensielt i følge Mary Welch og Paul R. Jackson \cite{Kommunikasjon} for å oppnå gode resultater for et gruppearbeid .
Planlegging var utfordrende der vi ikke hadde et system for å dele fremdriftsplanen som hver enkelt hadde. 
\\ \\
Det ble tidlig klart at dette var et problem, men ingen tok på seg ansvaret for å løse problemet eller tok eierskap til løsningen, noe som er viktig i følge “Teamwork in coopeartive extension programs”\cite{losning}. Dette er antagelig fordi vi hadde rullerende leder, med uklare ansvarsområder, som beskrevet i kapittel \ref{beslutning}.
Det at det ikke var en felles fremdriftsplan skapte irritasjon, så vel som inneffiktivitet.
 \\ \\
En annen faktor som var med å forsterke problemet var at vi ikke kjente hverandre godt i begynnelsen. Dette gjorde at tilbakemeldingene ofte var vage og det faktiske budskapet var tidvis godt innpakket. Etterhvert som vi ble bedre kjent, og så hvordan hvert gruppemedlem håndterte tilbakemeldinger og kritikk ble det også lettere for gruppemedlemmer å påpeke, samt korrigere ovenfor dem som ikke fulgte opp vedtak.
\\ \\
Det ble bestemt at gruppen skulle gå over til en mail løsning. Geir fikk ansvaret for å sette opp tjenesten, noe som sørget for at en fikk eierskap til problemet og ansvaret for gjennomføringen. Mail-tjenesten inneholdt muligheten til å sende mail til hele gruppen, samt en til en diskusjon mulighet. For gå bedre lagring av felles filer ble det besluttet å benytte Google Drive. 
Etter flere samtaler med Spark Ntnu ble det besluttet å benytte Google Sheet, med videre plan for å gå over til en platform som kan støtte en større organisasjon. 
Evaluering av tiltak
 \\ \\
Det å bruke mail fungerte bra som hoved informasjonskanal. Gruppemedlemmene fikk all nødvendig informasjon, noe som økte trivsel og effektivitet i gruppen.
 \\ \\
Det å benytte Google Sheet ga en fin oversikt over fremdriftsplanen til hver enkelt medlem. Siden vi har tenkt å bli en større organisasjon vil løsninge i fremtiden være uegnet. Vi ble anbefalt av Spark NTNU å gå over til Trello når vi har nådd en vis størrelse.