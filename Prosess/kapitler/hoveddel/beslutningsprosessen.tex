\documentclass[../main.tex]{subfiles}

\chapter{Vurdering av beslutningsprosessen}
	\label{beslutning}

De første EiT dagene etter at samarbeidsavtalen var skrevet, byttet vi hver landsbydag på hvem som skulle ha ansvaret for å lede gruppa. Dette gjorde vi for å at alle skulle få prøve seg i ulike roller og utforske komfortsonen sin, som er et av læringsmålene til EiT. Vi hadde også gått gjennom en personlighetstest for å bevisstgjøre hvilke roller hver av gruppemedlemmene hadde, men resultatet fra denne testen tok vi med en håndfull neve salt. (Ref til testen!!!!!!!!!!). Siden ingen i gruppen hadde erfaring med å bygge fjernstyrte undervannsroboter tidligere begynte alle bare å jobbe i hver sin ende uten å bli enige om en klar framgangsmåte eller et felles mål å jobbe mot. Et eksempel på dette var da Geir begynte å programmere før Morten og Kjetil hadde blitt enige om hvilken plattform de skulle bruke først. Her var gruppen altså ikke effektiv og gruppen opplevde situasjonen som krevende og kaotisk. Noe som til tider førte til tydelig frustrasjon og lav gruppemoral.
\\ \\
Det er viktig  for gruppen å nå sine mål ved å formulere mål og oppgaver som angår alle medlemmene \cite[kap 1]{jj}. Gruppen vår hadde i begynnelsen store vanskeligheter med å ta en rask og konkret beslutning. Ettersom vi byttet leder for hver landsbydag endret også målene seg, og kan ha noe med at vi hadde forskjellig oppfatning om hvordan vi ville gå frem for å løse oppgaven. En annen grunn kan ha vært at vi var veldig tidlig i gruppearbeidet og manglende kjennskap til hverandre som resulterte i usikkerhet, samt redsel for å komme med kritikk. Dette er en fase som kalles for, forming stage, i følge \cite{Tuckman65}.  Et tiltak vi gjorde for å bedre gruppens besluttsomhet og arbeidsflyt var å revidere samarbeidsavtalen. Den første endringen var å velge en klar leder som kunne styre oss i rett retning.  Siden det var Daniel som hadde startet prosjektet og de andre måtte søke seg inn var det naturlig for gruppa å velge Daniel som leder \cite[Storming]{Tuckman65}. En annen endring var å lage en liste over krav og funksjoner roboten skulle oppfylle for videre definere roller og ansvarsområder til hvert enkelt medlem i gruppa.
 \\ \\
Selv etter revidering av samarbeidsplanen var gruppemedlemmene fortsatt avhengig av at leder hadde oversikt over det som skulle gjøres, så vel som delegere ut oppgaver som skulle utføres Performing\cite{Tuckman65}. Dette var noe gruppen slet mest med gjennom hele prosjektet, og delegering av oppgaver og ansvarsområder fra leder kunne derfor vært bedre. Grunnen til dette kan ha vært at Daniel ikke var komfortabel med ansvaret og rollen som leder. Dette mener vi er viktig å ta med videre til senere gruppearbeid eller prosjekt. For selv om rollene og ansvarsområdene er satt, må en hele tiden passe på at alle får oppgaver som passer dem best. I følge Wheelan\cite{Susan} så vil gruppens prestasjoner lide når medlemmer blir tillagt roller eller oppgaver som ikke er passende. Vi merket at gruppens besluttsomhet og arbeidsflyt bedret seg utover prosjektet. Dette kan ha noe med at gruppen ble bedre kjent med hverandre, så vel som rollene og ansvarsområdene til hvert medlem ble klarere Norming \cite{Tuckman65}. For eksempel ble morgenmøtene kortere og mer effektive etter at Daniel fikk litt tid på seg til å sette seg inn i lederrollen. Dette førte til at gruppemoralen ble mye høyere enn tidligere og det ble mye mer motiverende å arbeide.
 \\ \\
Gjennom arbeidet med EIT har vi lært at det å ha flere ledere fungerer dårlig. Det skaper for mye kollektiv ansvar, noe som fører til mindre sannsynlighet for ansvarsfølelse for den enkelte gruppemedlem. Derfor besluttet vi å velge en klar leder som kunne ta beslutninger når gruppen ikke var enige og ikke kom seg framover. Lederen er en nøkkelrolle i en effektiv gruppe og det er viktig at medlemmene er klar over hva den innebærer og at de støtter den \cite{SchwarzA2002a}. For å støtte og utvikle lederrollen hadde vi etter hver EiT dag evaluering av leder med både positive og negative tilbakemeldinger.  
\\ \\
Vi fikk sett hvordan vi taklet tidspress og utfordringer knyttet til et prosjekt hvor det er veldig forskjellig sammensatt gruppe. Fra dette situasjonen er det mye lærdom vi kan ta med oss videre til andre gruppearbeid, og det er fortsatt mye vi kan bli bedre på. Lederen bør være strengere i å sette av tid til oppgaver, så vel som være mer bestemt i ledelsen for å få mer gjennomslag i beslutningene. Det kan være lurt å ha en jevnlig dialog mellom leder og hver enkelt gruppemedlem for å få frem uklarheter. Leder kan være bedre på å gi ærlige og konstruktive tilbakemeldinger som ikke er pakket inn i søte ord, dette gjelder også andre veien. Vi oppdaget også at selv om det er tidspress i et kompleks prosjekt er det alltid viktig at lederen hele tiden sørger for å koordinere og tydeliggjøre for gruppa hva planen er, samt delegere passende oppgaver som skal løses. Det kan spare mye tid og frustrasjon underveis. 






