\documentclass[../main.tex]{subfiles}

\chapter{Inneffektive møter}
	\label{c:moter}
Vi så fort at vi brukte mye tid på møter som ikke førte til en beslutning. De første to landsbydagene gikk mye av tiden til møter. Vi opplevde at vi ikke greide å holde oss til et tema under diskusjoner og det var mye tid som gikk med på å diskutere saker som ikke hadde mye med EIT å gjøre.  I løpet av møtene ble det ikke tatt mange beslutninger. Derfor måtte hver diskusjon tas flere ganger.
\\ \\
I starten av semesteret var møtene preget av mangel på struktur. Det var ingen overordnet plan over hvordan møte skulle foregå. Det som burde ha vært i denne planen er blant annet. Hvor lang tid møte skulle ta, hva møtet skulle omhandle og hvordan beslutningene skulle tas. Det at vi ikke visste hva møtet skulle omhandle før møtet hadde to negativ virkninger. Det ene var et en stor del av møtet gikk med til møteplanlegging og det andre var at  gruppemedlemmene ikke kunne forberede seg til temaene som ble diskutert.
\\ \\
For å få vekk dette problemet satte gruppen i gang et tiltak der hvert møte skulle ha en saksliste som alle gruppemedlemmene skulle ha lest før møtet startet. Det ble bestemt at møtelederen skulle ha ansvar å sende ut denne. Etter dette tiltaket ble antall temaer som ble diskutert mindre og møtene ble av den grunn korter og mer effektive. Grunnen til dette var blandt annet at det ble enklere å se om gruppen burde diskutere temaene før møtene startet slik at vi unngikk en situasjon der vi diskuterte et tema i lang tid og kom etter hvert til konklusjonen at dette ikke var et tema som var verdt å ta opp i plenum. Noe som definerte møtene de første dagene. Cohen\cite{Cohen} kommer fram til at det å publisere en møteplan før møtet gjennomføres og at alle deltakerne ser gjennom den, har positiv effekt på møtegjennomføringen. Dette stemmer godt overens med gruppes erfaring. 
\\ \\
Vi hadde problemer med å komme frem til besluttninger. Det kan komme fra at vi ikke visste hvordan vi skulle gå frem for å ta beslutninger. Dette gjorde at beslutningene ikke ble tatt og når beslutninger ble tatt ble de nødvendigvis ikke fulgt opp.
\\ \\
I samarbeidsavtalen ble det avtalt at beslutninger skulle tas med flertallbestemmelse, men gruppen hadde ikke fastsatt en øvre grense på hvor lenge en diskusjon kunne vare. Samarbeidsavtalen mangle elementer som tilatter at folk kunne skjære igjennom en diskusjon for å si at nå må en beslutning tas. En annen ting som kan ha ført til dette er at folk ikke følte ansvar for at møtet skulle føre frem og ansvar for at beslutningene skulle følges opp. Dette problemet mente gruppen ville løse seg selv ved at gruppen ble mer kjent med hverandre og samholdet økte og at medlemen i gruppen var obs på problemet. I følge Leana\cite{leana} vil medlemene i en gruppe med godt samhold være mer direkte i hvordan de diskuterer temaer og ikke sensurere seg selv under diskusjonen. Dette vil medføre at gruppemedlemmene ikke var redde for å komme med forslag til beslutninger og tiltak og vil være sikrere på sine egne forslag. En situasjon der forslag til tiltak blir diskutert mer fritt vil gi raskere og bedre beslutningsprosess, enn en situasjon der bare problemene blir diskutert, som var situasjonen i starten. Dessuten vil en gruppe der flest personer er involvert i beslutningsprosessen ifølge Schwarz\cite{SchwarzA2002a} være mer effektive. Da vil de fleste medlemen føle seg mer forpliktet til beslutningene og gruppen vil i følge teorien være mer effektiv.
Vi opplevde at medlemmene selvsensurerte seg mindre i diskusjoner jo lengre ut i prosjektet vi kom. Beslutningene ble også bedre fulgt opp mot slutten. Dette tyder på et viss samsvar mellom det vi opplevde og teori, men vi kan ikke være sikker på om valget våres og vente det ut var det korekte valget.

 \\ \\
Det å ha møteleder var et av punktene på teamkontrakten. Dette ble ikke fulgt opp første landsbydag etter teamkontrakten ble skrevet. Dette ble gruppen oppmerksom på da vi evaluerte dagen og vi bestemte oss for å skaffe en møteleder til neste møte. Det neste møtet ble ikke veldig mye bedre selv om gruppen hadde en møteleder. Problemet her kan ha vært at gruppen ikke hadde definert oppgaver til møtelederen og at medlemene i gruppen som ikke hadde fått rolllen som møteleder, forventet at møtelederen alene skulle sørge for at møtene blir bedre. Wheelan\cite{susan} skriver at effektive team-medlemmer har i oppgave å ta ansvar for fasilitering i stedet for å skylde på lederen. Rollen møtelederen har består av å  lage liste over ting som skal diskuteres under møtet og å sørge for at alle får denne før møtet startet, sette av tidsbruk til møtet og å sørge for at dette blir følgt opp og å styre ordet hvis det er behov for det. 
Etter møtelederens oppgaver ble bedre definert og at møtelederen utførte disse oppgavene, gikk tidsbruken på møtet ned og de andre deltakerene fikk større motivasjon til å komme forberedt til møtet og deltok i fasitliteringen i en større grad.

Vi hadde problemer med å finne et rom der vi kunne jobbe på landsbydagene. Dette kommer av at det er mange EIT-grupper som har lysst på rom på onsdager. Vi prøvde å være ut i god tid med rombestillingen, men det beste vi fikk til var å bestille rom for en halv dag. Da vi ikke fikk tilfredsstillende rom på ntnu, bestemte vi oss for å jobbe i leiligheten til to av medlemmene. I følge Cohen\cite{cohen} vil et bra møtelokale være et med få distraksjoner. Det var en del distraksjoner hjemme hos de to studentene. Så vi konkluderte med at dette ikke var den beste måten å løse romprobematikken på.
