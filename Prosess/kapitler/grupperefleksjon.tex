\documentclass[../main.tex]{subfiles}
\chapter{Grupperefleksjon}

Vortex NTNU har i løpet av vår semesteret jobbet sammen og utviklet seg fra å være en forhåndsbestemt sammensetning av individer, til en gruppe med en felles oppfattelse og forståelse, dynamikk og følelse. Gjennom de forskjellige prosessøvelsene, så vel som under det faglige prosjektarbeidet har gruppen modnet. Åpen kommunikasjon og gjensidig respekt har stått sentralt i denne prosessen. Vi har lært en god del om problemstillinger og utfordringer ved å jobbe i et tverrfaglig team i et stort prosjekt. Gruppemedlemmene har opplevd at det til tider har vært veldig mye å gjøre og at det ikke har vært mangel på ambisjonsnivå ved målsetningen for prosjektet. \\ \\
Vi har lært at det å bli kjent med en ny gruppe mennesker hvor medlemmene var litt varsomme ovenfor hverandre. Tuckman (1965, i Johnson & Johnson, 2006) kaller denne fasen for forming stage, som er fasen med usikkerhet i gruppen hvor medlemmene gjør seg kjent med reglene innad i gruppen og sin egen rolle.
 \\ \\
Tuckmans teori klassifiserer gruppeprosessen i ulike utviklingsfaser; fra formingsfasen, over stormings-, normerings- og prestasjonsfasen, til avslutningsfasen, der gruppen fullfører arbeidet og går i oppløsning. Sett i retrospektiv har vi fulgt denne teorien tett uten å vite det underveis. Etter hvert som vi ble bedre kjent, åpnet vi oss mer og tillot oss mer konstruktiv kritikk i både selv- og grupperefleksjon, samt åpen dialog og debatt. (24)
 \\ \\
Vi hadde grupperefleksjoner hver dag som skulle bidra til at samarbeidet ble bedre. Der diskuterte vi problemer som vi så med samarbeidet og kom med aksjoner som kunne forbedre sammarbedeidet. Forslagene til aksjoner ble stort sett fattet i konsensus. Grunnen til dette var muligens at gruppen var preget av konfliktskyhet og at vi manglet andre forslag. Det at alle i gruppen var veldig motivert til både å forbedre samarbeidet og å få til et godt prosjekt, førte til at alle var veldig positive til forslagene som kom og det kan være en av grunnene til at mange av aksjonene ble fattet med konsensus. Likevel burde det være litt mer diskusjoner og uenigheter, siden gruppen er satt sammen av personer med forskjellige bakgrunner og personligheter. Selv med disse problemene mener vi at gruppen oppdaget og diskuterte mange problemer med samarbeidet og at vi kom fram til nyttige aksjoner.
 \\ \\
Noen aksjoner vi tenkte å gjennomføre i begynnelsen gikk til tider dessverre i glemmeboken og førte til en viss grad av ineffektivitet fordi vi ble nødt til å revurdere prosesser på nytt for å finne nye og bedre løsninger til våre utfordringer. Selv om det til tider føltes kjedelig i prosjektprosessen hjalp disse revurderingene av tidligere satte aksjoner oss til å evaluere og forbedre arbeidet. Det fikk oss også til å se klare mønstre på hvilke problemer vi fortsatt slet med. Imidlertid hadde dette den positive effekt at vi fikk prøvd ut flere forskjellige tilnærminger til problemløsninger som vi kan ta med oss videre til senere gruppearbeider. Siden vi ikke var sikker på hvilke regler og normer som ville forbedre effektiviteten har vi kanskje satt oss regler og føringer som ikke alltid var like bra for effektiviteten i gruppen, noe Wheeler (2009) prøver å advare mot. Vi forstår at det som fungerer i en gruppe, nødvendigvis ikke fungerer i alle grupper og at prøving og feiling kan være nødvendig for å finne de beste løsningene i forskjellige grupper.
 \\ \\
Vi hadde problemer med å legge plan for hvordan vi skulle gå frem for å  nå målene som vi hadde satt for dette prosjektet. Dette kom stort sett fra at vi ikke hadde oversikt over hvilke prosesser som vi måtte gjennom, og hvor lang tid de forskjellige delmålene ville ta. Etter at vi har vært gjennom dette prosjektet mener vi selv at vi har bygget opp den erfaringen vi manglet i starten, slik at hvis vi hadde begynnt på prosjektet nå ville vi ha gjort en mye bedre jobb. 
 \\ \\
Sammarbeidsindikatoren til gruppen har forbedret seg dramatisk på punktet effektiv og strukturert. Dette samsvarer godt med det gruppen selv opplever. For eksempel ble vi bedre på å se prosjektet i sin helhet noe vi hadde en tendens til å ikke gjøre i starten. De fleste aksjonene hadde målet å føre til bedre effekt og disse resultatene peker mot at disse aksjonene ga positivt resultat.
 \\ \\
Vi har gjennom dette prosjektet lært at alle grupper er forskjellige. Et godt utgangspunkt for en effektiv gruppe er at en må man ha en felles forståelse for hvilke mål man vil nå og hvordan man skal nå det målet. Utfordringer som dukker opp underveis i arbeidet må tas opp så snart som mulig for å unngå at arbeidet sklir ut. Det å ha klare sett med regler og  fellesforståelse for hvordan gruppen skal fungere bidrar til å forbedre kontakten innad i gruppen og skaper dermed et fortrolig og godt miljø for diskusjon og debatt.Når gruppemedlemmene er trygge på hverandre er det mye lettere å diskutere på en god og effektiv måte, konstruktiv og med gjensidig respekt. 
 \\ \\
Vi har sett på noen felles personlighetskarakteristikker som vi mener har vært med på å forme gruppen. Vi har alle vært interessert i oppgaven, samarbeidsvillig, åpen for å lære av og med hverandre og hatt en generell positiv innstilling til EiT. Dette har ført til godt samarbeid og god stemning. Vi har også alle vært litt konfliktsky, noe vi mener har ført til litt trege beslutningsprosesser, men bedre gruppekohesjon og mindre friksjon. Vi tar med oss videre at det kan være lurt å si fra litt tidligere for at gruppen skal komme seg raskere fremover. Vi har også sammen vurdert hvordan hverandres personlighetskarakteristikker har påvirket gruppesamarbeidet. Morten har vist stor arbeidskapasitet og vært åpen for å gi og ta imot tilbakemeldinger. Dette har inspirert resten til å legge inn et ekstra arbeid. Geir har brukt sin kreativitet til å få gruppen til å se på prosjektet fra nye vinkler. Han har heller ikke vært redd for å ta på seg ekstra oppgaver når han har hatt best kunnskap om disse. Helge har vert med på å bidra til at vi tidlig fikk en god stemning i gruppen. Han har også vært en pådriver for å holde arbeidet i gang samt påpeke kritiske feil og mangler i samarbeidet. Daniel har blitt mer komfortabel med lederrollen og har ført arbeidet fremover med en tydelig plan.


