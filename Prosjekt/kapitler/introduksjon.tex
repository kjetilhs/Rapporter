\documentclass[../main.tex]{subfiles}

\section{Introduksjon}
Vi vil i denne rapporten vise våre resultater fra Byggelandsbyen i Eksperter i Team (EiT) våren 2015. Vårt prosjekt har vært å designe og bygge en fjernstyrt undervannsrobot, en såkalt ROV. Formålet med dette var å bygge kompetanse til å kunne danne en organisasjon parallelt med EiT, slik at denne kan delta i MATE ROV konkurransen i USA sommeren 2016. Det ble derfor besluttet at vi skulle prøve  å holde oss innenfor MATE ROV konkurransens spesifikasjoner samtidig som vi måtte overholde sponsorenes krav til design.

\subsection{Hva er en ROV?}
En ROV er en fjernstyrt, ubemannet undervannsfarkost (Remotely Operated Vehicle) som får strøm via en kabel fra et skip. ROVen har flere propeller montert slik at den kan gå framover, sidelengs og snu om sin egen akse. Kommunikasjon mellom ROV og piloten om bord på skipet går også via denne kabelen. Video, sonarbilder og andre sensordata blir sendt opp fra ROVen, og piloten styrer den med en joystick. Mange ROVer har manipulatorarmer for å gjøre arbeid under vann. Disse styres av piloten eller en co-pilot. ROVer kan utstyres med mye energikrevende utstyr siden de får strøm fra skipet, og de kan sende opp video og data via en fiber i ROV-kabelen.

ROV brukes til å inspisere og arbeide under vann. Med en sterk og presis manipulator kan den erstatte dykkere i mange arbeidsoperasjoner. ROVen kan også gå mye dypere. Arbeid på flere tusen meters dyp er mulig, og ROVer er helt uunnværlige for installasjoner av utstyr for utvinning av olje og gass på dypt vann.

\subsection{MATE ROV Konkurransen}
MATE ROV konkurransen er en unik designkonkurranse som ble startet av Marine Advanced Technology Education (MATE) senter. Industrien opplevde problemet at dagens studenter manglet praktisk kunnskap, samarbeidsevner og prosjektledelse. Derfor startet de MATE for å gi elevene en plattform hvor de kan utvikle disse ferdighetene.
 
MATE Konkurransen er basert på følgende: Elevene må tenke på seg selv som ''entreprenøre´´ og forvandle laget sitt til selskaper som produserer, markedsfører, og selger ''produktene´´sine. I tillegg til prosjektering, er studentene pålagt å utarbeide tekniske rapporter, postere, og tekniske presentasjoner levert til konkurransens dommerpanel bestående av fagfolk fra industrien.
 
Elevene blir utfordret til å designe og bygge en ROV som skal konkurrere mot andre lag fra hele verden. Oppdragene som skal gjennomføres er modellert etter scenarier fra havet.
Konkurransens varierende vanskelighetsgrad med nybegynner, middels og avansert gir studentene mulighet til å anvende sine ferdigheter og kunnskap uansett utdanningsnivå. Etter hvert studentene konstruerer stadig mer avanserte ROVer for komplekse oppgaver, blir det større rom for elevene å bygge videre på ferdighetene sine. 


\subsection{Detaljer om konkurransen}
Denne unike konkurransen er ikke vunnet utelukkende av teamet som fullfører oppdragene raskest, men ved å bevise at laget ditt er bedre med design, teamorganisering, og økonomisk og salgsplanlegging. Denne filosofien er godt reflektert i utformingen av konkurransen.

Under konkurransen er det mulig å tjene 600 poeng totalt, fordelt på de ulike aspekter av konkurransen. Laget med høyest poengsum vinner.

Konkurransen er delt inn i fire klasser; Explorer, ranger, navigator and scout class

Vortex NTNU har som mål å være i explorer class som har følgende poengfordeling som vist i Figur \ref{f:mate}.

\begin{table}[h!]
  \begin{center}
  \caption{Poenggiving i MATE ROV}
  \label{f:mate}
    \begin{tabular}{l l}
    Hva gir poeng & Hvor mye poeng [poeng]\\
    \hline
    Oppdrag 						& 300 \\
    Sikkerhet 						& 10 \\
    Organisasjonens effektivitet 	& 10 \\
    Tidsbonus 						& Varierer \\
    Technical Documentation 		& 100 \\
    Sales Presentation 				& 100 \\
    Marketing Display				& 50 \\
    Safety inspection				& 30 \\
    \hline
    totalt 							& 600 + tillegg\\
    \hline
    
    \end{tabular}
  \end{center}
  
\end{table}


